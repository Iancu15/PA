\documentclass[runningheads]{llncs}
\usepackage{graphicx}
\usepackage{amsmath}
\usepackage{amsfonts}
\usepackage[none]{hyphenat}
\usepackage{enumitem}
\usepackage{comment}
\usepackage{pifont}
\usepackage{listings}
\usepackage{xcolor}
\lstset { %
	language=C++,
	backgroundcolor=\color{black!5}, % set backgroundcolor
	basicstyle=\footnotesize,% basic font setting
}
\begin{document}
\lstset{
	tabsize = 2,
}
\title{Tema 0}
\author{Iancu Alexandru-Gabriel 324CD}

\institute{University Politehnica of Bucharest \\ 
	Faculty of Automatic Control and Computers}
\maketitle

\section{Divide et Impera - Floor of Sorted Array}
\subsection{Enunt}
Pentru un vector sortat se cauta cea mai mare valoare mai mica sau egala cu o valoare data la intrare. Rezultatul
este indexul din vector al valorii gasite, daca nu este gasita valoarea se intoarce -1.
\subsection{Descrierea solutiei}
Ideea principala este folosirea cautarii binare prin vector pana gasim valoarea cautata, acest lucru ne este permis
intrucat vectorul este sortat. Ne orientam in functie de cum se pozitioneaza x fata de inceputul, sfarsitul si mijlocul
vectorului de la iteratia curenta:
\begin{enumerate}
	\item Daca primul element este mai mare decat x, atunci nu exista elemente mai mici decat x care sa satisfaca conditia
si se intoarce -1 (daca se va ajunge la acest caz, se va observa de la prima iteratie);
	\item Daca ultimul element este mai mic sau egal decat x, atunci el este sigur cel mai mare element din vector care
sa satisfaca conditia si se va intoarce ultima pozitie din vectorul curent;
	\item Daca valoarea de mijloc este mai mare decat x:
	\begin{enumerate}
		\item Daca valoarea de dinainte de mijloc este mai mai mica sau egala cu x, atunci acela este rezultatul si se intoarce
mid - 1;
		\item Altfel se merge recursiv pe vectorul din stanga (exclusiv) mijlocului pentru ca toate elementele de la mijloc incolo
sunt mai mari decat x si de neinteres.
	\end{enumerate}
	\item Altfel daca valoarea de mijloc este mai mica sau egala decat x:
	\begin{enumerate}
		\item Daca valoarea de dupa mijloc este mai mare decat x, atunci inseamna ca mijlocul este elementul cautat;
		\item Altfel se merge recursiv pe vectorul din dreapta (exclusiv) mijlocului pentru ca toate elementele de dinainte de
mijloc sunt mai mici decat x si de neinteres pentru ca ne-am asigurat valoarea de la mid + 1 care este sigur mai mica sau egala
decat x si mai mare decat toate celelalte elemente din stanga.
	\end{enumerate}
\end{enumerate}
Cum este cautare binara, recursivitatea va fi mereu pe o singura ramura din cele 2 optiuni prezente. Astfel in momentul in care
o iteratie se termina si intoarce ceva, se va termina tot programul.
\subsection{Algoritmul de rezolvare}
\begin{lstlisting}[
basicstyle=\small
]
floorSearch(arr[], low, high, x)
	if (arr[low] > x or low > high)
		return -1

	if (arr[high] <= x)
		return high
	
	mid = (low + high) / 2
	
	if (arr[mid] > x)
		if (arr[mid - 1] <= x)
			return mid - 1
		else
			return floorSearch(arr, low, mid - 1, x)
	
	if (arr[mid + 1] > x)
		return mid
	else
		return floorSearch(arr, mid + 1, high, x)
\end{lstlisting}
\vspace*{1em}
In al treilea if nu trebuie verificat ca mid $>$ 0 (pentru arr[mid - 1]), pentru ca mid = 0 doar pentru:
	\begin{enumerate}
	\item (low = 1, high = 0), motiv in care nu s-ar ajunge acolo pentru ca ar intra in primul if ca respecta
conditia low $>$ high;
	\item (low = 0, high = 1), caz in care se poate discerne rezultatul uitandu-ne doar la orientarea lui x
fata de arr[low] si arr[high].
	\end{enumerate}
\subsection{Complexitate si eficienta}
	\begin{enumerate}
	\item In cadrul unei iteratii se fac doar operatii de O(1) pe langa apelurile recursiv. $\implies$ f(n) = O(1)
	\item La urmatoarea iteratie se introduce un vector cu un numar de elemente injumatatit fata de cel initial. $\implies$ b = 2
	\item Intotdeauna se va merge pe o singura ramura o data conform unei implementari obisnuite de cautare binara. $\implies$ a = 1
\end{enumerate}
Astfel algoritmul prezentat are urmatoarea relatie de recurenta:

\begin{align*}
T(n) = T(\frac{n}{2}) + \theta(1)
\end{align*}

Care se poate rezolva folosind teorema Master:

\begin{gather*}
n^{\log_b a} = n^{\log_2 1} = n^{0} = 1 \\
f(n) = \theta(1) \in theta(n^{0}\log_2^{0} n) = \theta(1)\text{, }k = 0 \\
\implies \text{conform cazului 2 al teoremei Master } T(n) \in \theta(\log_2 n)
\end{gather*}

Complexitatea rezultata este cea asteptata de la un algoritm ce foloseste cautarea binara (ce are complexitate de $\theta(\log_2 n)$) si
care are complexitate O(1) la fiecare iteratie recursiva. Algoritmul din punct de vedere al complexitatii temporale este optim pentru
ca foloseste cautarea binara intr-un vector deja sortat(este mediul de lucru optim pentru a folosi cautarea binara). Din punct de vedere
al complexitatii spatiale, daca luam in considerare stiva de executie, este de O($log_2 n$) pentru ca toate cele $log_2 n$ apeluri recursive
vor fi stocate pe stiva. Motivul pentru care este O in loc de $\theta$ este pentru ca if-urile din cadrul unei iteratii pot oferi rezultatul
prematur si astfel se poate ajunge la mai putine apeluri recursive. Din punct de vedere spatial se poate optimiza pentru a ajunge la O(1)
folosind varianta iterativa a cautarii binara, astfel nemaiavand nevoie de stiva pentru stocarea de apeluri.
\subsection{Exemplu}
Pentru exemplu iau vectorul [1 3 7 8 15 20 23 30] si voi incerca folosind algoritmul de mai sus sa caut floor-ul lui 18:
\begin{itemize}
	\item low = 0 high = 7:
	\begin{itemize}
		\item vectorul iteratiei - [1 3 7 8 15 20 23 30]
		\item mid = $\frac{0 + 7}{2}$ = 3 (am facut rotunjire in jos)
		\item arr[mid] = arr[3] = 8 $<$ 18, iar arr[3 + 1] = 15 $<$ 18 $\implies$ se merge recursiv pe a doua jumatate
		\item se actualizeaza low = mid + 1 = 4, high ramane la fel
	\end{itemize}

	\item low = 4 high = 7:
	\begin{itemize}
		\item vectorul iteratiei - [15 20 23 30]
		\item mid = $\frac{4 + 7}{2}$ = 5
		\item arr[mid] = arr[5] = 20 $>$ 18, iar arr[5 - 1] = 15 $<$ 18 $\implies$ 15 este floor-ul
		\item se intoarce mid - 1 = 5 - 1 = 4
	\end{itemize}
\end{itemize}

\section{Greedy - Egyptian Fraction}
\subsection{Enunt}
Reprezentati o fractie subunitara primita la intrare ca o fractie egipteana. O fractie egipteana e un mod de a reprezenta fractiile folosind doar
unitati fractionare. O unitate fractionara este o fractie cu numaratorul 1 si numitorul un numar natural pozitiv. Spre exemplu
$\frac{3}{4}$ poate fi reprezentat ca $\frac{1}{4}$ + $\frac{1}{2}$.
\subsection{Descrierea solutiei}
Pentru a afla prima si cea mai mare unitate fractionara luam fractia initiala si o simplificam cu numaratorul. Rezultatul va fi o unitate fractionara
cu numitorul un numar zecimal. In acest moment cautam fractia mai mica sau egala cu pseudo unitatea noastra fractionara calculata, cum ambele
au numaratorul 1, este destul sa comparam numitorul celor 2. Daca pentru fractia per total cautam un numar mai mic sau egal, la numitor e invers si astfel
cautam un numar mai mare sau egal cu numarul zecimal calculat si anume primul astfel de numar intreg. Apoi pentru a gasi a doua unitate fractionara
calculam noua fractie care va fi restul ramas din scaderea primei unitati fractionare din fractia initiala, se aplica apoi aceeasi pasi. Se tot continua
in aceasta maniera pana cand numitorul calculat este un numar intreg, fractia simplificata rezultata fiind ultima unitate fractionara.
\subsection{Algoritmul de rezolvare}
\begin{lstlisting}[
basicstyle=\small
]
egyptianFraction(nr, dr)
	fraction = nr/dr
	unit_dr = ceil(1/fraction)
	
	while ((1/fraction) is not an integer)
		print("1/" + unit_dr + " ")
		fraction -= 1/unit_dr
		unit_dr = ceil(1/fraction)
	
	print("1/" + (1/fraction) + " ")
\end{lstlisting}
\vspace*{1em}
Se calculeaza fractia initiala si apoi se face ceiling (rotunjire in sus) asupra inversei fractiei. Dupa se afiseaza unitatea fractionara si
se scade din fractia curenta, iar mai apoi se calculeaza numitorul pentru unitatea fractionara. Se continua in acest fel pana cand
inversa fractiei este un numar intreg. Dupa ce se iese din while se afiseaza ultima unitate fractionare. Functia ciel se considera ca
functioneaza conform rotunjirii in sus matematice si anume urmatorul numar intreg mai mare sau egal numeric cu acesta.

Un mod de a verifica daca inversa fractiei nu este un numar intreg este sa testam daca numarul se incadreaza intr-o anumita toleranta primita
ca parametru:

\begin{align*}
abs(\frac{1}{fraction} - 10^{-i}) > 0\text{, unde i este magnitudinea tolerantei}
\end{align*}

O problema totusi cu implementarea de mai sus este ca e imprecisa pentru ca e dependenta de precizia calculatorului si cum numerele fractionare
pot avea un sir infinit de cifre, prin rotunjire se va pierde din exactitate. Varianta de pe pagina urmatoare rezolva problema
\newpage
\begin{lstlisting}[
basicstyle=\small
]
egyptian_fractionsv2(nr, dr)
	unit_dr = ceil(dr/nr);
	nr = nr * unit_dr - dr;
	dr *= unit_dr;
	
	while (dr > 0)
		print("1/" + unit_dr + " ")
		unit_dr = ceil(dr/nr);
		nr = nr * unit_dr - dr;
		dr *= unit_dr;
\end{lstlisting}
\vspace*{1em}
Se poate rezolva acest lucru prin calcularea la fiecare pas a numaratorului si a numitorului care isi vor pastra valoarea fiind numere intregi
nerestrictionate de precizie. Conform noului algoritm la fiecare pas intai se calculeaza numaratorul unitatii fractionare ca in varianta anterioara,
iar mai apoi se actualizeaza numitorul si numaratorul fractiei curente. Se tot face acest lucru pana cand numaratorul este 0 si anume s-a
epuizat toata fractia. Legat de calcularea numitorului si numaratorului, se aduce fractia anterioara si unitatea fractionara la 
acelasi numitor amplificand ambele fractii cu numitorul celeilalte:

\begin{align*}
\frac{nr}{dr} - \frac{1}{unit\_dr} = \frac{nr * unit\_dr}{dr * unit\_dr} - \frac{dr}{dr * unit\_dr} = \frac{nr * unit\_dr - dr}{dr * unit\_dr}
\end{align*}

Varianta a doua functioneaza exact, insa este limitata de dimensiunea maxima a variabilelor nr, dr si unit$\_$dr.
\subsection{Complexitate si analiza proprietatilor}
Complexitatea temporala este O(u), unde u este egal cu numarul de unitati fractionare, pentru ca se efectueaza 3 operatii O(1) pentru fiecare unitate
fractionara calculata (functia ciel se considera ca are complexitate O(1) fiind o simpla rotunjire in sus). Nu se poate determina numarul de unitati
fractionare ale unei fractii fara ca aceasta sa fie procesata in prealabil. Complexitatea spatiala este O(1) daca lucram in octeti, daca luam numarul de biti
in calcul este O(log n). Insa si la aceasta problema si la cea precedenta si la urmatoarele o sa calculez complexitatea spatiala in octeti.

Algoritmul este de tip Greedy, asa ca el calculeaza optimul local si anume unitatea fractionara pentru fractia curenta in fiecare pas al problemei.
Se ia mereu cea mai mare unitate fractionara si cum orice fractie poate fi reprezentata ca o suma de unitati fractionare, atunci se poate spune ca
algoritmul va ajunge mereu la optimul global prin optimele locale. Dupa ce se calculeaza o unitate fractionara, se sustrage aceasta unitate
fractionara din fractia curenta, rezultatul fiind fractia urmatoarei iteratii. Astfel dupa calcularea optimului local se satisface calcularea
corecta a urmatorului optim local si implicit al optimului global $\implies$ se respecta proprietatea alegerii locale.

Solutiile optime ale subproblemelor sunt unitatile fractionare, iar solutia problemei este multimea acestor unitati fractionare care este
unica pentru orice fractie data la intrare. Optimul global al problemei este o multime a rezultatelor optimelor locale si astfel solutia problemei
contine si solutiile subproblemei $\implies$ se respecta proprietatea de substructura optima.
\subsection{Exemplu}
Pentru fractia $\frac{4}{13}$ folosind varianta 2 avem urmatorii pasi:
\vspace*{1em}
\begin{itemize}
	\setlength\itemsep{1em}
	\item Calcularea primei unitati fractionare:
	\vspace*{0.2em}
	\begin{itemize}
		\setlength\itemsep{0.6em}
		\item se calculeaza numitorul fractiei unitate ca fiind $ciel(\frac{13}{4}) = 4$ $\implies$ prima unitata fractionara este $\frac{1}{4}$
		\item numaratorul noii fractii este $4 * 4 - 13 = 16 - 13 = 3$
		\item numitorul noii fractii este $13 * 4 = 52$
	\end{itemize}
	\vspace*{1em}
	\item Fractia curenta: $\frac{3}{52}$ Unitatile fractionare: $\frac{1}{4}$
	\vspace*{0.2em}
	\begin{itemize}
		\setlength\itemsep{0.6em}
		\item numitorul fractiei unitate - $ciel(\frac{52}{3}) = 18$ $\implies$ a doua unitate fractionara este $\frac{1}{18}$
		\item numaratorul - $3 * 18 - 52 = 54 - 52 = 2$
		\item numitorul - $52 * 18 = 936$
	\end{itemize}
	
	\item Fractia curenta: $\frac{2}{936}$ Unitatile fractionare: $\frac{1}{4}$ $\frac{1}{18}$
	\vspace*{0.2em}
	\begin{itemize}
		\setlength\itemsep{0.6em}
		\item numitorul fractiei unitate - $ciel(\frac{936}{2}) = 468$ $\implies$ a treia unitate fractionara este $\frac{1}{468}$
		\item numaratorul - $2 * 468 - 936 = 936 - 936 = 0$ $\implies$ s-a epuizat fractia initiala, s-au descoperit toate unitatile fractionare
	\end{itemize}
\end{itemize}

Rezultatul este: $\frac{1}{4} + \frac{1}{18} + \frac{1}{468}$

\section{Programare dinamica - nth Catalan Number}
\subsection{Enunt}
Pentru un numar natural n primit la intrare, se doreste sa se calculeze al n-lea numar catalan. Numerele catalane apar intr-o multitudine de
probleme de combinatorica. Spre exemplu numarul de arbori binari de cautare posibili cu N chei este echivalent cu al N-lea numar catalan.
Astfel se poate adapta un algoritm de calculare a numerelor catalane pentru toate problemele de numarare aferente, scopul principal
fiind descoperirea unui algoritm eficient de calculare a acestor numere pentru a putea rezolva eficient si celelalte probleme de
interes.
\subsection{Descrierea solutiei si relatia de recurenta}
Numerele catalane respecta urmatoarea relatie de recurenta cunoscuta:

\begin{equation}
C_{n}=
\begin{cases}
1, & \text{daca}\ n=0 \\
\sum_{i=0}^{n-1} C_{i}*C_{n-1-i}, & \text{daca}\ n \in \mathbb{N}^*
\end{cases}
\end{equation}

Cum fiecare numar catalan e calculat in functie de toate numerele catalane precedente, programarea dinamica se prezinta drept un mod
eficient de rezolvare a problemei. Prin metoda memorizarii se pot stoca toate numerele catalane calculate anterior pentru a fi folosite
ulterior, vom folosi un vector pentru asta.
\subsection{Algoritmul de rezolvare}
\begin{lstlisting}[
basicstyle=\small
]
nth_catalan_number(n)
	catalan = new vector[n + 1]
	catalan[0] = 1;
	for (i = 1; i <= n; i++)
		catalan[i] = 0
		for (j = 0; j < i; j++)
			catalan[i] += catalan[j] * catalan[i - j - 1]

	return catalan[n]
\end{lstlisting}
\vspace*{1em}
Se initializeaza cazul de baza si anume al 0-lea numar catalan care este egal cu 1. Apoi pentru fiecare numar catalan urmator pana la n
se calculeaza suma conform index-ului aferent si al numerelor catalane precedente.
\\
\subsection{Complexitate}
Algoritmul are 2 treceri imbricate de complexitate O(n), asa ca algoritmul are complexitate temporala de O($n^{2}$), unde n este egal
cu index-ul numarului catalan cautat. Complexitatea spatiala este O(n + 1) = O(n) pentru ca se foloseste un vector care salveaza
al n-lea numar catalan si toate numere precedente necesare pentru calcularea acestuia de la al 0-lea la al (n - 1)-lea.
\subsection{Exemplu}
Pentru exemplificarea algoritmului, il voi folosi pentru a calcula al 4-lea numar catalan:
\begin{itemize}
	\setlength\itemsep{1em}
	\item catalan = [1 0] index = 1 (index-ul numarului catalan ce urmeaza sa fie calculat la pasul curent)
	\begin{itemize}
		\setlength\itemsep{0em}
		\item se iau indexi de la 0 la 0
		\item pentru indexul 0 se adauga catalan[0] * catalan[1 - 0 - 1] = 1 $\implies$ catalan[1] = 0 + 1 = 1
	\end{itemize}

	\item catalan = [1 1 0] index = 2
	\begin{itemize}
		\setlength\itemsep{0em}
		\item se iau indexi de la 0 la 1
		\item pentru indexul 0 se adauga catalan[0] * catalan[2 - 0 - 1] = catalan[1] = 1 $\implies$ catalan[2] = 1
		\item pentru indexul 1 se adauga catalan[1] * catalan[2 - 1 - 1] = 1 * 1 = 1 $\implies$ catalan[2] = 1 + 1 = 2
	\end{itemize}

	\item catalan = [1 1 2 0] index = 31
	\begin{itemize}
		\setlength\itemsep{0em}
		\item se iau indexi de la 0 la 2
		\item pentru indexul 0 se adauga catalan[0] * catalan[3 - 0 - 1] = catalan[2] = 2 $\implies$ catalan[3] = 2
		\item pentru indexul 1 se adauga catalan[1] * catalan[3 - 1 - 1] = 1 $\implies$ catalan[3] = 2 + 1 = 3
		\item pentru indexul 2 se adauga catalan[2] * catalan[3 - 2 - 1] = 2 $\implies$ catalan[3] = 3 + 2 = 5
	\end{itemize}

	\item catalan = [1 1 2 5 0] index = 4
	\begin{itemize}
		\setlength\itemsep{0em}
		\item se iau indexi de la 0 la 3
		\item pentru indexul 0 se adauga catalan[0] * catalan[4 - 0 - 1] = catalan[3] = 5 $\implies$ catalan[4] = 5
		\item pentru indexul 1 se adauga catalan[1] * catalan[4 - 1 - 1] = catalan[2] = 2 $\implies$ catalan[4] = 5 + 2 = 7
		\item pentru indexul 2 se adauga catalan[2] * catalan[4 - 2 - 1] = 2 $\implies$ catalan[4] = 7 + 2 = 9
		\item pentru indexul 3 se adauga catalan[3] * catalan[4 - 3 - 1] = 5 $\implies$ catalan[4] = 9 + 5 = 14
	\end{itemize}
\end{itemize}
Al 4-lea numar cautat va fi catalan[4] si anume 14.
\section{Backtracking - Word Break Problem}
\subsection{Enunt}
Se doreste sa se imparta un sir de caractere fara spatii, primit la intrare, intr-un sir de cuvinte despartite de spatii. Dictionarul ce
contine cuvintele posibile poate fi primit la intrare sau hardcodat in cazul in care se doreste ca dictionarul sa fie constant de la o
rulare la alta. Algoritmul va afisa toate posibilitatile de impartire a sirului de caractere folosind dictionarul. Exemplu:
\begin{itemize}
	\item Dictionar = ["testare" "test" "are" "testament" "ament" "ghetari"]
	\item Input = "testaretestament"
	\item Output:
		\begin{itemize}
			\item test are test ament
			\item test are testament
			\item testare test ament
			\item testare testament
		\end{itemize}
\end{itemize}
\subsection{Descrierea solutiei}
Se primeste sirul ce urmeaza sa fie prelucrat la intrare. Pentru acest sir se vor crea toate prefixele nenule posibile, spre exemplu daca sirul
are dimensiune n, atunci o sa avem prefixul care contine primul caracter, cel care contine primele doua ... cel care contine toate caracterele
din sir. Pentru fiecare astfel de sir se va verifica daca acesta face parte din dictionar, daca nu face atunci este ignorant. Cand se gaseste
un prefix care satisface conditia, acesta este adaugat la rezultatul pentru combinatia respectiva si se continua prelucrarea restului de sir.
Se tot adauga prefixe la rezultat si se tot prelucreaza siruri intermediare pana cand fie nu a fost un gasit niciun prefix care sa respecte
conditia la un anumit pas, fie s-a ajuns la finalul sirului si au fost descoperite toate cuvintele din sir care apartin dictionarului. In ambele
cazuri se face backtracking si se intoarce la pasul intermediar precedent pentru a se cauta urmatoarele combinatii.
\subsection{Algoritmul de rezolvare}
\begin{lstlisting}[
basicstyle=\small
]
wordBreakUtil(str, n, result, dictionary[])
	for (i = 1; i <= n; i++)
		prefix = str.substring(0, i);

		if (dictionary.contains(prefix))
			if (i == n)
				print(result + prefix)
				return

			rest_of_str = str.substring(i, n)
			new_result = result + prefix + " "
			wordBreakUtil(rest_of_str, n - i, new_result, dictionary)

wordBreak(str, dictionary[])
	wordBreakUtil(str, str.size(), "", dictionary)
\end{lstlisting}
\vspace*{1em}
Se incepe algoritmul cu intregul sir si drept rezultat un sir gol ce urmeaza sa fie marit de cuvintele aferente. In prima 
prelucrare se vor extrage prefixele folosind functia substring(start, i) care selecteaza primele i caractere de la al start-lea
caracter(indexarea sirului este de la 0). Apoi folosind functia contains(str) se va cauta prefix-ul str in dictionar, daca este gasit se
intoarce adevarat si se continua iteratia curenta. In interior-ul if-ului daca prefixul e fix intregul sir curent inseamna ca s-a gasit o
combinatie si este afisat rezultatul stocat alaturi de sirul curent verificat. Insa daca este oricare prefix, atunci se
mai face un apel al functiei pentru prelucrarea restului de sir, adaugand prefix-ul curent la rezultat.

Pentru ca algoritmul se foloseste de apeluri recursive, rezultatele intermediare vor fi stocate pe stiva recursiva.
\newpage
\subsection{Complexitate si eficienta}
Pentru fiecare dimensiune i posibila de la 1 la n a prefixelor, se va calcula pentru fiecare dimensiune j posibila de la 1 la (n - i)
ce va calcula pentru fiecare dimensiune k posibila de la 1 la (n - i - j) s.a.m.d., unde n este lungimea sirului. Suma tuturor acestor
calcule poate fi reprezentata astfel:

\begin{align*}
n * ((n - 1) * ((n - 2) * ((n - 3) * ((n - 4) * ..) + ..) + ..) + ..)
\end{align*}

Doar luand in calcul una din numeroasele inmultiri avem n * (n - 1) * (n - 2) * .. 1 ce ne da o complexitate de O($n^n$), se poate
astfel afirma ca tot algoritmul are complexitate de O($n^n$) si astfel are o complexitate exponentiala. Desigur nu se mereu trece prin
toate posibilitatile pentru ca ignoram prefixele invalide, insa in cel mai rau caz si anume dictionarul este format din toate submultimile
de caractere posibile ale intrarii, se vor considera mereu prefixele valide si astfel se va trece prin toate posibilitatile posibile(acesta
fiind worst case scenario).

Functia substring copiaza i caractere din sirul iteratiei curente si astfel are complexitate O(n). Cautarea in dictionar are
complexitate O(n * w), unde w este numarul de cuvinte din dictionar pentru ca se presupune ca foloseste o comparare caracter cu
caracter intre siruri, iar sirul comparat este un prefix ce are dimnensiunea maxima n pentru cazul in care dictionarul este un simplu
vector de siruri. Astfel complexitatea de facto a algoritmului este O($2 * w * n^2 * n^n$) = O($w * n^n$) pentru ca pentru fiecare prefix
sunt maxim 2 apeluri de substring(o data pentru extragerea sa din sir, a doua oara in cazul in care este valid pentru extragerea restului de sir)
si un apel de cautare in dictionar.

Complexitatea spatiala a algoritmului este O(n) pentru ca pot sa fie maxim n apeluri recursive si variabilele aferente prezente pe stiva,
acest lucru intamplandu-se cand dictionarul are tot alfabetul sirului primit si astfel se apeleaza pentru cate un caracter o data.

Problema nu poate fi rezolvata mai eficient pentru ca acesta necesita sa afiseze toate posibilitatile si astfel programarea
dinamica nu poate fi folosita. Ce se poate eficientiza sunt functiile folosite in interiorul algoritmului. Functia de cautare in dictionar
ar putea fi eficienta daca ar cauta in timp ce ar compara intr-un vector de siruri sortat folosind cautarea binara, astfel ar cauta
intai dupa primul caracter, apoi dupa al doilea s.a.m.d. sau dictionarul ar putea fi reprezentat ca un hash map cu o functie de hash
eficienta cu putine coliziuni.
\subsection{Exemplu}
O sa folosesc exemplul prezentat in enunt ce avea dictionarul ["testare" "test" "are" "testament" "ament" "ghetari"] si intrarea
"testaretestament":
\begin{itemize}
	\setlength\itemsep{1em}
	\item sirul curent = "testaretestament"
	\begin{itemize}
		\setlength\itemsep{0em}
		\item se cauta prefixele "t", "te" si "tes" in dictionar si nu sunt gasite
		\item se gaseste "test" in dictionar, se actualizeaza rezultatul $\implies$ "test"
		\item se merge pe restul sirului care este "aretestament"
	\end{itemize}

	\item sirul curent = "aretestament"
	\begin{itemize}
		\setlength\itemsep{0em}
		\item prefixele "a" si "re" nu se regasesc in dictionar
		\item "are" se regaseste $\implies$ "test are"
	\end{itemize}

	\item sirul curent = "testament"
	\begin{itemize}
		\setlength\itemsep{0em}
		\item se gaseste "test" in dictionar $\implies$ "test are test"
	\end{itemize}

	\item sirul curent = "ament"
	\begin{itemize}
		\setlength\itemsep{0em}
		\item se gaseste "ament" in dictionar $\implies$ "test are test ament"
		\item s-a ajuns la finalul sirului, se afiseaza rezultatul si se face backtracking
	\end{itemize}

	\item sirul curent = "testament"
	\begin{itemize}
		\setlength\itemsep{0em}
		\item se gaseste "testament" in dictionar $\implies$ "test are testament"
		\item s-a ajuns la final, se afiseaza rezultatul
	\end{itemize}
	
	\item sirul curent = "aretestament"
	\begin{itemize}
		\setlength\itemsep{0em}
		\item nu se mai gaseste alt sir in dictionar, ne intoarcem la sirul precedent
	\end{itemize}
	
	\item sirul curent = "testaretestament"
	\begin{itemize}
		\setlength\itemsep{0em}
		\item se gaseste "testare" $\implies$ "testare"
		\item se merge pe restul sirului "testament"
	\end{itemize}
	
	\item sirul curent = "testament"
	\begin{itemize}
		\setlength\itemsep{0em}
		\item se gaseste "test" in dictionar $\implies$ "testare test"
	\end{itemize}

	\item sirul curent = "ament"
	\begin{itemize}
		\setlength\itemsep{0em}
		\item se gaseste "ament" in dictionar $\implies$ "testare test ament"
		\item s-a ajuns la finalul sirului, se afiseaza rezultatul si se face backtracking
	\end{itemize}
	
	\item sirul curent = "testament"
	\begin{itemize}
		\setlength\itemsep{0em}
		\item se gaseste "testament" in dictionar $\implies$ "testare testament"
		\item s-a ajuns la final, se afiseaza rezultatul
	\end{itemize}

	\item sirul curent = "testaretestament"
	\begin{itemize}
		\setlength\itemsep{0em}
		\item nu se mai gaseste alt prefix in dictionar
		\item se inchide programul
	\end{itemize}
\end{itemize}

Output-ul este identic cu cel prezentat in enunt.

\section{Analiza comparativa}
\subsection{Divide et impera}
Este o tehnica de programare prin care o problema se imparte in 2 sau mai multe subproblemele care vor fi
evaluate separat si combinate ulterior, aceste subprobleme impartindu-se la randul pana se ajunge la un caz
de baza. Astfel problemele pentru care aceasta tehnica poate fi aplicata sunt cele care pot fi impartite in
subprobleme de aceeasi natura ca problema initiala.

Avantajul acestei tehnici este ca poate calcula eficient acest gen de probleme. De asemenea prin faptul ca
se imparte in mai multe subprobleme, ea poate fi paralelizata pentru a le calcula simultan. Cum dimensiunea
problemelor scade exponential, atunci algoritmul paralel ar avea per total o complexitate logaritmica. 
Dezavantajul este ca se foloseste de recursivitate si ocupa stiva de executie, astfel dimensiunea problemei este limitata de
dimensiunea stivei. Daca se poate folosi greedy sau programare dinamica pentru a se rezolva problema,
este recomandat sa se foloseasca acestea deoarece tind sa fie mai eficiente. Pe de alta parte algoritmii
divide et impera sunt mai eficienti decat backtracking in rezolvarea problemelor ce necesita interogarea
recursiva de parti ale problemei initiale.
\subsection{Greedy}
O tehnica ce functioneaza pe problemele care respecta proprietatea alegerii locale si proprietatea de substructura
optima. Proprietatea alegerii locale ne spune ca prin calcularea optimului local al problemei se ajunge la
optimul global. Cea de-a doua proprietate restrictioneaza solutia problemei sa fie o multime formata din
solutiile subproblemelor. Asa ca problemele care pot fi rezolvate prin aceasta tehnica trebuie sa respecte cele
doua proprietati sau cel putin proprietatea alegerii locale(cum e problema rucsacului fractionara).
De asemenea in general greedy e o alegere buna pentru problemele care doresc calcularea de extreme
si anume numarul maxim sau minim care respecta o anumita conditie.

Avantajul acestei tehnici este eficienta pentru ca proceseaza o singura data elementele. Intrarile care nu
satisfac cerintele algoritmului pot fi prelucrate in prealabil, in general folosindu-se o sortare a datelor ce are complexitate
de O(n * log n), acest lucru in cazul anumitor probleme se considera un dezavantaj. Daca problema se rezolva
folosind un algoritm greedy in O(n), iar la aceasta se adauga o prelucrare de O(n * log n), tot algoritmul
va fi de fapt O(n * log n).

Daca exista totusi un algoritm de programare dinamica care rezolva in O(n), este
de preferat sa fie folosit acesta pentru ca nu este supraincarcat de o prelucrare in prealabil. Insa daca
problema functioneaza cu greedy fara prelucrare, deseori este o varianta mai eficienta fata de programarea
dinamica pentru ca tehnica greedy nu este interesata de detalii ca programarea dinamica si asa calculeaza
doar operatiile necesare.
\subsection{Programare dinamica}
Programarea dinamica se foloseste pentru rezolvarea problemelor pentru care s-a observat sau se stie o relatie de
recurenta. Asa ca pentru ca problemele sa poata fi rezolvate folosind aceasta tehnica, trebuie gandita in prealabil
relatia de recurenta, daca o asemenea relatie nu exista sau e foarte greu de conceput nu se poate aplica
programare dinamica. Algoritmele se folosesc de metoda memorizarii conform careia se stocheaza toti pasii
calculati din relatia de recurenta pana se ajunge la elementul din sir dorit.

Avantajul vizibil al tehnicii este eficienta foarte mare pentru ca stocheaza toate valorile necesare intr-un
vector pentru a nu le calcula ulterior, sarind astfel peste anumite calcule redundante. Alt avantaj este faptul
ca rezolva astfel eficient si problemele care doresc la iesire toate valorile din sirul relatiei de recurenta
pana la final fara a se adauga un cost la timp sau spatiu consumat. Un dezavantaj evident este faptul ca se
foloseste de spatiu pentru a compensa pentru eficienta in timp, asa ca pentru a folosi metoda memorizarii
va avea mereu complexitati de cateva magnitudini mai mari decat O(1), comparativ cu Greedy care nu memorizeaza
pasii intermediari si astfel poate sa aiba complexitate spatiala de O(1).

Alt dezavantaj este ca e greu paralelizabil pentru ca este restrans de relatia de recurenta. Daca pentru
fiecare urmator element din relatia de recurenta este nevoie de cel calculat imediat precedent, atunci algoritmul
este obligat sa fie liniar. In retrospectiva daca problema se poate rezolva cu divide et impera este probabil
sa se ajunga la o eficienta timp mai mare daca se adapteaza aceasta la paralelism.
\subsection{Backtracking}
Backtracking este o tehnica ce se foloseste pentru problemele ce necesita toate combinatiile posibile aferente
unei intrari sau care doresc sa se gaseasca prima combinatie care respecta o anumita conditie. Face acest
lucru folosindu-se de operatia de backtracking care intoarce algoritmul de fiecare data la pasul precedent in urma
intalnirii indeplinirii conditiei de succes sau de fail.

Avantajul tehnicii este ca e cea mai buna optiune pentru prelucrarea de permutari. Dezavantajul este ca
generarea de permutari duce deseori la o complexitate exponentiala, insa daca asta doreste problema este
o metoda mai buna de a face asta. In comparatie metoda naiva este mult mai ineficienta pentru ca aceasta
ia toate cazurile chiar daca nu toate sunt de interes, prin metoda backtracking oprindu-se
cautarea intr-o directie cand se indeplineste conditia de fail si se continua pe celelalte cazuri.

Daca cumva problema nu necesita toate permutarile sau nu trebuie trecute
prin toate pentru a se ajunge la o solutie, atunci nu este recomandat sa se foloseasca backtracking, desi
merge, din cauza eficientei scazute. In asemenea situatii se recomanda sa se foloseasca ceilalti algoritmi
prezentati in functie de tipul problemei. In cazul in care de fapt este vorba de un sir ce satisface o
relatie de recurenta este de preferat sa se foloseasca programarea dinamica. Backtracking-ul este recomandat
doar daca nu exista nicio alta tehnica ce ar putea rezolva problema.
\subsection{Exemple}
\subsubsection{Problema rucsacului}
Programarea dinamica reuseste sa ofere un algoritm de complexitate
O(n * w), unde n este numarul de obiecte si w capacitatea rucsacului. Pe de alta parte, problema
s-ar mai putea rezolva folosind backtracking, insa astfel s-ar ajunge la o complexitate exponentiala. Sunt
multe probleme care s-ar putea rezolva folosind backtracking, insa trebuie intai sa ne gandim daca problema
apartine P, iar daca da sa gasim o tehnica potrivita ce ne ofera un algoritm cu timp polinomial.
In situatia problemei rucsacului, algoritmul ce foloseste programarea dinamica in cel mai rau caz poate ajunge 
la o complexitate pseudo exponentiala in cazul
in care capacitatea rucsacului este foarte mare, totusi in majoritatea cazurilor este de multe magnitudini mai
bun decat varianta cu backtracking.

\subsubsection{Nth ugly number}
Problema calcularii numerelor "urate", numere a caror divizori sunt doar 2, 3 sau 5,
are o metoda ce foloseste programarea dinamica si reuseste sa calculeze in timp O(n). Totusi de remarcat este faptul ca
se poate folosi si cautarea binara, lucru ce ne duce la o complexitatea temporala de O(log n). Astfel programarea
dinamica are dezavantajul, in anumite situatii, ca nu poate sa produca o complexitate mai buna decat O(n), pe cand
divide et impera reuseste sa atinga O(log n) folosind cautarea binara. De asemenea in aceasta situatie cautarea
binara este o alegere mai buna pentru ca nu foloseste spatiu auxiliar pe langa stiva de executie.

Asa ca daca
problema se poate face folosind doar o singura cautare binara(sau un numar constant de cautari binare), atunci
nu merita sa cauti o relatie de recurenta
pentru un algoritm cu programare dinamica pentru ca e putin probabil sa o faca mai bine. In general programarea dinamica si
greedy sunt bune pentru ca fac o singura trecere prin sirul dat la intrare sau cel ce trebuie creat treptat, insa astfel
sunt constransi in a avea deseori o complexitate de maxim O(n).

\subsubsection{Word Wrap Problem}
Pentru un sir de caractere se doreste sa se puna newline astfel incat cuvintele sa ramana intregi, se primeste la
intrare dimensiunea maxima a unei linii. Prima metoda si cea mai simpla este greedy care pur si simplu ia fiecare
cuvant pe rand si umple fiecare linie cat se poate ca mai apoi sa treaca la linia urmatoare. Metoda functioneaza
bine in majoritatea cazurilor, insa pentru anumite cazuri nu balanseaza cum trebuie spatiile pentru ca ii pasa
doar de optimul local, nu de intreaga panorama. Algoritmul este totusi destul de eficient, trecand doar o data
prin toate cuvintele are o complexitate de O(n) si este inca folosit de MS Word si OpenOffice.org.

Totusi problema mai poate fi rezolvata folosind programare dinamica si anume relatia de recurenta prezentata in
cartea lui Cormen. Aceasta foloseste o matrice si astfel umplerea acesteia produce o complexitate temporala a algortimului de O($n^2$).
Algoritmul este optim in toate cazurile, insa nu este la fel de rapid ca si Greedy. A trebuit dat la schimb
eficienta pentru corectitudine schimband Greedy cu programarea dinamica. In multe probleme va trebui sa facem
aceasta alegere in functie de ce ne intereseaza cu adevarat sau ce intereseaza firma. Daca o simpla divergenta
de la optimalitate ar duce la mari erori in sistem, atunci ne permitem sa dam la schimb eficienta. Altfel
daca nu e asa de important, atunci mai bine folosim ce se executa mai repede. In problema prezentata optimalitatea
nu e asa importanta, nu conteaza asa mult cat de balansate sunt spatiile, conteaza in principal ca liniile sa se incadreze
in dimensiunea precizata pastrand intregimea cuvintelor. Viteza si simplitatea sunt motivele pentru care metoda Greedy inca este folosita in procesoarele word.
\begin{thebibliography}{6}
	\bibitem{divide_et_impera}
	Floor in a Sorted Array, \\
	\url{https://www.geeksforgeeks.org/floor-in-a-sorted-array/} \\
	\bibitem{greedy}
	Greedy Algorithm for Egyptian Fraction, \\
	\url{https://www.geeksforgeeks.org/greedy-algorithm-egyptian-fraction/} \\
	\bibitem{programare_dinamica}
	Program for nth Catalan Number, \\
	\url{https://www.geeksforgeeks.org/program-nth-catalan-number/} \\
	\bibitem{catalan_number}
	Catalan number, \\
	\url{https://en.wikipedia.org/wiki/Catalan$\_$number} \\
	\bibitem{backtracking}
	Word Break Problem using Backtracking, \\
	\url{https://www.geeksforgeeks.org/word-break-problem-using-backtracking/} \\
	\bibitem{rucsac}
	0-1 Knapsack Problem | DP-10, \\
	\url{https://www.geeksforgeeks.org/0-1-knapsack-problem-dp-10/} \\
	\bibitem{ugly}
	Ugly Numbers, \\
	\url{https://www.geeksforgeeks.org/ugly-numbers/} \\
	\bibitem{wrap}
	Word Wrap Problem | DP-19, \\
	\url{https://www.geeksforgeeks.org/word-wrap-problem-dp-19/} \\
	\bibitem{springer}
	LNCS Homepage, \url{http://www.springer.com/lncs}.  \\

	Toate link-urile au fost accesate ultima data pe 25 aprilie 2021.
\end{thebibliography}
\end{document}